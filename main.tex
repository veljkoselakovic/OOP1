\documentclass{article}
\usepackage[utf8]{inputenc}
\usepackage[english,serbian]{babel}
\usepackage[fleqn]{mathtools}
\usepackage{marginnote}
\usepackage{tikz}
\usepackage{relsize}
\usepackage{enumitem}
\usepackage{wasysym}
\usepackage{multicol}
\usepackage{listings}
\usepackage{color}
\usepackage[none]{hyphenat}
\usepackage[bottom = 0.5in]{geometry}
\usepackage{changepage}

% Definisanje lstset za formatiran kod
\definecolor{dkgreen}{rgb}{0,0.6,0}
\definecolor{gray}{rgb}{0.5,0.5,0.5}
\definecolor{mauve}{rgb}{0.58,0,0.82}
\lstset{frame=tb,
  language=C++,
  aboveskip=3mm,
  belowskip=3mm,
  showstringspaces=false,
  columns=flexible,
  basicstyle={\small\ttfamily},
  numbers=none,
  numberstyle=\tiny\color{gray},
  keywordstyle=\color{blue},
  commentstyle=\color{dkgreen},
  stringstyle=\color{mauve},
  breaklines=true,
  breakatwhitespace=true,
  tabsize=3
}

%Definisanje okoline za naslove sa vecim znacenjem - jako debilno
\newenvironment{xitemize}{%
    
    \itemize
    \larger
}{%
    \enditemize
}
\let\olditemize\itemize
\let\endolditemize\enditemize
\renewenvironment{itemize}{%
    \smaller
    \olditemize
}{%
    \endolditemize
}
% Kod za definisanje checkmarka
\def\checkmark{\tikz\fill[scale=0.4](0,.35) -- (.25,0) -- (1,.7) -- (.25,.15) -- cycle;} 
% Kod za definisanje inline koda
\providecommand{\inlinecode}[1]{\texttt{#1}}
% Kod za komandu koja uokviri tekst i na margini napise dodatnu informaciju u formatu \boxedimportant[info]{text}
\providecommand{\boxedimportant}[2][]{\noindent\fbox{%
            \parbox{\linewidth}{%
            #2            }%
                        \marginnote{#1}

            }
}
\addtolength{\topmargin}{-.875in}

\title{OOP1\\\large Objektno-orijentisano programiranje 1}
\author{Veljko Selaković }
\date{prof. dr Igor Tartalja\\ prof. dr Dragan Milićev}


\begin{document}

\maketitle
Ako nadjete greske recite mi odmah da ispravim
Nebitno da li su slovne ili sam nesto pogresno lupio. Mozete i sami izmeniti na https://github.com/veljkoselakovic/OOP1 i napraviti merge request
\newpage

\begin{itemize}
    \item \textbf{UML} - \textit{Unified Modeling Language} (Kasnije predmet \textit{Projektovanje softvera}, 5. semestar)
\end{itemize}
\section{Osnovni ciljevi OOP}
\begin{xitemize}
    \item Problem korišćenja postojećeg koda
    \begin{itemize}
        \item \textbf{Biblioteka funkcija} - skupo održavanje, otklanjanje grešaka i proširivanje sistema
    \end{itemize}
    \item Evolucija programskih jezika
    \begin{enumerate}
        \item Apstrakcija izraza  $\sim$1950. - \textbf{FORTRAN}
        \begin{itemize}
            \item Registri skriveni
        \end{itemize}
        \item Apstrakcija kontrole $\sim$1960. - \textbf{Algol60}
        \begin{itemize}
            \item Tok kontrole programa - \textit{petlje}
        \end{itemize}
        \item Apstrakcija podataka  $\sim$1970. - \textbf{Pascal}
        \begin{itemize}
            \item Razdvajanje detalja prezentacije podataka od apstraktnih operacija koje se definišu nad podacima - npr. \textit{tipovi nabrajanja}
        \end{itemize}
    \end{enumerate}
    \item Dodatni koncepti
    \begin{enumerate}
        \item Zasebno prevođenje modula - \textbf{FORTRAN, C, Ada}
        \item Razdvajanje interfejsa od implementacije - \textbf{Ada}
        \item Koncept klase - \textbf{Simula67}
    \end{enumerate}
    \item 4 Osnovna principa OOP
    \begin{itemize}
        \item Apstrakcija
        \item (En)kapsulacija
        \item \underline{Nasleđivanje}
        \item Polimorfizam
    \end{itemize}
\end{xitemize}
\section{C++}
\begin{xitemize}
    \item Razvoj C++
    \begin{itemize}
        \item  C $\rightarrow$ C sa klasama $\rightarrow$ C++
        \item Svake 3 godine novi standard
        \item ISO 98 $\rightarrow$ ISO 03 $\rightarrow$ ISO 11 $\rightarrow$ ISO 14 $\rightarrow$ISO 17 $\rightarrow$ ISO 20 \textit{(još nije standardizovano)}
        \item Spontani razvoj, za razliku od \textbf{Ade}
    \end{itemize}
    
    \item Aspekti C++
    \begin{enumerate}
        \item Da bude dovoljno \underline{blizak mašini}
        \begin{itemize}
            \item C++ je \textit{nadskup} u velikoj većini slučajeva
        \end{itemize}
        \item Da bude dovoljno \underline{blizak problemu}
        \begin{itemize}
            \item Klase iz \textbf{Simula67}
            \item Preklapanje operatora iz Algola
        \end{itemize}
    \end{enumerate}
\end{xitemize}
\newpage
\section{Pregled gradiva koje će se raditi}
\begin{xitemize}
    \item Klase i objekti
    \begin{itemize}
        \item Klase su apstrakcije zajedničkih atributa i zajedničkog ponašanja jednog skupa srodnih objekata
        \item Klasa sadrži
        \begin{enumerate}
            \item Podatke članove \textit{(atribut ili polje)}
            \item Funkcije članove \textit{(metodi)}
        \end{enumerate}
        \item Pristupačnost određenim članovima deklariše programer
        \item \textbf{Implementaciju} klase čine $\longleftarrow$ \textit{Kako radi?}
        \begin{enumerate}
            \item Privatni podaci članovi
            \item Definicije funkcija
        \end{enumerate}
        \item \textbf{Interfejs} klase čine $\longleftarrow$ \textit{Šta radi?}
        \begin{enumerate}
            \item Javni podaci članovi
            \item Deklaracije javnih funkcija 
        \end{enumerate}

        \item Instanca klase $\rightarrow$ objekat $\begin{cases}
        \textbf{Stanje}\\
        \textbf{Ponašanje}\\
        \textbf{Identitet}
        \end{cases}$
    \end{itemize}
    \item Konstruktori i destruktori
    \begin{itemize}
        \item Prilikom kreiranja i uništavanja objekta
        \item Nemaju \inlinecode{return} tip
        \item Automatsko izvršavanje prilikom kreiranja/uništavanja objekta
    \end{itemize}
    \item Izvođenje i nasleđivanje
    \begin{itemize}
        \item Iz opštije klase izvodimo specifične klase
        \item Izvedene klase nasleđuju atribute i metode osnovne klase, i dodaju nove
        \item Objekti izvedene klase su i indirektne instance osnovne klase
        \item U izrazima izvedeni objekti mogu zameniti osnovnu klasu - \textit{Liskov substitution principle}
        \item Nasleđeni metodi se mogu redefinisati
    \end{itemize}
    \item Polimorfizam
    \begin{itemize}
        \item Ako se funkcija proglasi \inlinecode virtual na nju se primeni \textbf{\underline{dinamičko vezivanje}}
        \item \textbf{\underline{Dinamičko vezivanje}} - Adresa se ne određuje u vreme povezivanja, poziv se vezuje za funkciju u vreme izvršenja
        \item Ponašanje objekta ne zavisi samo od tipa pokazivača, već i od tipa pokazanog objekta
    \end{itemize}
    
    \item Klasifikacija objektnih jezika
    \begin{enumerate}
        \item Objektno-bazirani
       \begin{itemize}
           \item Apstrakcija, (en)kapsulacija, modularnost\\ \textbf{Ada83, Visual Basic 6}
       \end{itemize} 
       \item Objektno-orijentisani
       \begin{itemize}
           \item Princip nasleđivanja\\
           \textbf{Simula, Smalltalk, Ada95, C++, Java, VB.Net, C\#}
       \end{itemize}
    \end{enumerate}
    \item Obrada izuztenih situacija
    \begin{itemize}
        \item Nepostojeće datoteke, prekoračenje opsega indeksa,...
        \item Tradicionalni jezici u funkciji vraćaju vrednost koja signalizira grešku, koja se naknadno analizira
        \item Kod postane nepregledan
        \item   \inlinecode{try/catch/throw} ključne reči
    \end{itemize}
    \item Šabloni \textit{(Templates)}
    \begin{itemize}
        \item Određene obrade ne zavise od tipa podataka
        \item \textit{Generičko programiranje}
        \item Statički mehanizam - \textit{u prevodu se zameni}
    \end{itemize}
    \item Preklapanje operatora
    \begin{itemize}
        \item Sam koncept nije OO, ali se dobro uklapa
        \item Redefinicija standardnih jezičkih operatora
        \\   \inlinecode{operator<simbol>}
        \item Ne mogu se preklopiti svi operatori, a za neke važe posebna pravila
    \end{itemize}
\end{xitemize}

\section{Proširenja jezika C}
\begin{xitemize}

    \item Deklaracija vs definicija
    \begin{multicols}{2}
    \begin{itemize}
    
        \item Deklaracija je iskaz koji
        \begin{enumerate}
            \item Uvodi ime u program
            \item Govori prevodiocu kojoj jezičkoj kategoriji pripada ime
        \end{enumerate}
        \item Definicija
        \begin{enumerate}
            \item Kreira objekat ILI
            \item Navodi telo funkcije ILI
            \item U potpunosti navodi strukturu korisničkog tipa
        \end{enumerate}
        
    
    \end{itemize}
    \columnbreak
    \begin{lstlisting}
    void f(int x, float y);       // Deklaracija
    void f(int x, float y) {...} // Definicija
    extern int x; // Deklaracija
    int x; //Definicija
    class X; //Deklaracija
    class X { ... }; //Definicija
    \end{lstlisting}
    \end{multicols}
    \begin{itemize}
        \item Samo jedna definicija, a proizvoljno mnogo deklaracija
        \item Objekat može biti definisan i deklarisan u istom redu
    \end{itemize}
    \item Objekti
    \begin{itemize}
        \item Objekat u širem smislu - \textbf{podatak}
        \item Objekat u užem smislu - \textbf{instanca klase}
        \item Objekat ima 
        \begin{enumerate}
            \item Stanje
            \item Ponašanje
            \item Identitet $\rightarrow$ primitivni podaci nemaju identitet
        \end{enumerate}
        \item \textbf{Promenljiva} je lokacija u kojoj se čuva podatak
        \item Podela promenljivih
        \begin{enumerate}
            \item Statička
            \item Automatska
            \item Dinamička
            \item Privremena \textit{(tranzijentna)}
        \end{enumerate}
    \end{itemize}
    \begin{multicols}{2}
    \item Lvrednosti \textit{(LVALUE)}
    \begin{itemize}
        \item Izraz koji upućuje na objekat ili funkciju
        \item Operatori čiji operandi moraju biti LVAL \\
        \textbf{unarni  \inlinecode{\&, ++, --,} levi operandi svih operatora dodele}
        \item Operatori čiji su rezltati LVAL\\
        \textbf{unarni   \inlinecode{*, [],} prefiksni   \inlinecode{++} i   \inlinecode{ --,} operatori dodele}
        \item DVrednost je sve što nije LVrednost \textit{(RVALUE)}
    \end{itemize}
    \columnbreak
    \begin{lstlisting}
    int *q[100]; 
    q[10]=&i;  //q[10] je lvrednost
    *q[10]=1; // *q[10] je lvrednost 
    q = &i; // ERROR, ime niza nije vrednost
    int a=1, b=2, c=3;
    (a=b)=c; // OK
    (a+b)=c; // ERROR
    ++ ++i; // OK
    i++ ++; // ERROR
    \end{lstlisting}
    \end{multicols}


\newpage
    \item Oblast važenja
    \textit{(Scope)}
    \begin{itemize}
        \item Onaj deo teksta programa u kome se deklarisano ime može koristiti
        \item \textbf{\underline{Dinamičko vezivanje}} imena $\rightarrow$ od mesta deklaracije do kraja datoteke (globalna imena)
        \item \textbf{Lokalna} imena $\rightarrow$ od mesta deklarisanja do kraja odgovarajućeg bloka
        \item Sakrivanje imena 
        \begin{enumerate}
            \item  Ako se definiše u nekom bloku, globalno je skriveno
            \item \boxedimportant[BITNO]{Ako se redefiniše u unutrašnjem bloku, ime iz spoljašnjeg bloka je sakriveno do izlaska iz bloka}
        \end{enumerate}
        \item Pristup globalnom imeu koristeći operator  \inlinecode{::}
    \end{itemize}
    \begin{lstlisting}
    int x = 0; // Globalno x
    void f() {
        int y=x, x; // y dobija vrednost globalnog x
        x=1; // Lokalno x
        ::x=5; // Globalno x
        {
            int x; // Novo lokalno x, sakriva prethodno
            x=2;
        }
        x=3; // Pristup prvom lokalnom
    }
    int *p = &x; // Globalno x
    \end{lstlisting}


    \item Specifični dosezi
    \begin{itemize}
        \item Oblast važenja funkcije imaju samo labele
        \item  \inlinecode{for} petlja
        \item Ranije verzije kompajlera su imale doseg promenljivih koji je bio 1 blok \textbf{van} (  \inlinecode{for} (MS VC++.6)
        \item U  \inlinecode{if} doseg do kraja  \inlinecode{else}
    \end{itemize}
    \item Klasni/Strukturni doseg
    \begin{itemize}
        \item Oblast važenja imaju svi njeni članovi
        \begin{enumerate}
            \item  \inlinecode{.} $\rightarrow$ levi operand objekat
            \item   \inlinecode{->} $\rightarrow$ levi operand pokazivač na objekat
            \item  \inlinecode{::} $\rightarrow $ levi operand ime klase
        \end{enumerate}
    \end{itemize}
    \item Životni vek objekata
    \begin{itemize}
        \item Vreme u toku izvršavanja programa u kojem objekat postoji i za koje mu se može pristupati
        \item Vek atributa klase = vek objekta
        \item Vek parametra = vek automatskog objekta
        \begin{enumerate}
            \item Statički objekti
            \item Automatski objekti
            \item Dinamički objekti
            \item Privremeni \textit{(tranzijentni)} objekat
        \end{enumerate}
    \end{itemize}
    \item Statički i automatski objekti
    \begin{itemize}
        \item Automatski objekat je lokalni objekat koji nije definisan kao \textit{static}
        \begin{enumerate}
            \item \textbf{Životni vek} - od definicije do kraja oblasti važenja
            \item Svaki put se kreira iznova prilikom poziva bloka u kom je definisan
            \item Prostor se alocira na \textit{stack}
        \end{enumerate}
        \item Statički objekat je globalni objekat ili lokalni deklarisan kao \textit{static}\\\\
        Globalni 
        $\begin{cases}
        $1. \textbf{Životni vek} - od definicije do kraja izvršenja \textit{main}$\\
        $2. Kreiraju se jednom, na početku izvršavanja, pre funkcija objekta$\\
        \end{cases}$\\
        Lokalni\hspace{0.26cm}$\begin{cases}
        $3. Počinju da žive pri prvom nailasku na njih$
        \end{cases}$
    \end{itemize}
    \newpage
    \item Dinamički i privremeni objekti
    \begin{itemize}
        \item Dinamički objekti se kreiraju i uništavaju posebnim operacijama
        \begin{enumerate}
            \item \textbf{Životni vek} - kontroliše programer
            \item   \inlinecode{new/delete}
            \item Prostor alocira na \textit{heap}
        \end{enumerate}
        \item Privremeni objekti se kreiraju pri izračunavanju izraza
        \begin{enumerate}
            \item \textbf{Životni vek} - kratak i nedefinisan
            \item Odlaganje međurezultata i privremeno smeštanja vraćenih vrednosti funkcije
            \item Najčešće se uništavaju čim nisu potrebni
        \end{enumerate}
    \end{itemize}


    \item Leksički elementi
    \begin{itemize}
        \item Komentari
        \begin{enumerate}
            \item \inlinecode{$//...$}
            \item \inlinecode{$/*...*/$}
        \end{enumerate}
        \item 73 ključne reči + alternative
        \item C++11 \inlinecode{bool}, C11:  \inlinecode{\_Bool}
        \item Specijalne (ali ne i rezervisane) reči: \inlinecode{final} i  \inlinecode{override} \textit{(Backwards compatibility)}
        \item Ne treba započinjati imena donjom crtom
    \end{itemize}

\item Tipizacija
\begin{itemize}
    \item \textbf{Stroga tipizacija} - objekti različitih tipova se ne mogu proizvoljno zamenjivati
    \item C++ je hibridan
    \begin{enumerate}
        \item Za osnovne primitivne tipove \checkmark{}
        \item Za sve ostalo X
    \end{enumerate}
\end{itemize}
\item Konverzija
\begin{itemize}
    \item Operatori zahtevaju određene tipove operanada
    \item Naredbe zahtevaju određene tipove operanada \\
    \inlinecode{for, if, while...}
\end{itemize}
\item Vrste konverzija tipova
\begin{itemize}
    \item \textbf{Standarna konverzija} - ugrađeno u jezik\\
    npr. \inlinecode int $\rightarrow $ \inlinecode float, \inlinecode char $\rightarrow$ \inlinecode int
    \item \textbf{Korisnička konverzija} - definiše programer
    \item Pored toga, konverzija može biti
    \begin{enumerate}
        \item \textbf{Implicitna} - prevodilac je automatski vrši
        \item \textbf{Eksplicitna} - zahteva programer
    \end{enumerate}
    \item C koristi \textit{cast} operator \\
    \inlinecode{(tip)izraz}
    \item C++ uvodi 4 specifčna \textit{cast} operatora
    \item Postoji i konverzioni konstruktor
\end{itemize}
\item Priduživanje imena tipu
\begin{multicols}{2}

\begin{itemize}
    \item C-stil\\
    \inlinecode{typedef opis\_tipa = ime\_tipa}
    \item C++ stil, čitljivije\\
    \inlinecode{using ime\_tipa = opis\_tipa}
\end{itemize}
\columnbreak
    \begin{lstlisting}
typedef unsigned long long int Ceo1;
using Ceo2 = unsigned long long int; 
    \end{lstlisting}
\end{multicols}
\begin{multicols}{2}

\item Određivanje tipa izrazom\newline
\inlinecode{decltype (izraz) promenljiva [= vrednost]}
\begin{itemize}
    \item Izraz se \textbf{ne} izračunava
    \item Primena kod \textit{template-ova}
\end{itemize}
\columnbreak
\begin{lstlisting}
int x=1; double y=2.3;
decltype(x) a = x; // a je int
decltype(y) b = y; // b je float
decltype(a++) c = a; // c == a == 1
\end{lstlisting}
\end{multicols}
\item Automatsko određivanje tipa
\begin{itemize}
    \item \inlinecode{auto} ključna reč određuje tip na osnovu inicijalne vrednost 
    \inlinecode U C, \inlinecode{auto} označava automatsku lokalnu promenljivu, i ne piše se
    
    \begin{lstlisting}
auto int a = 10; // Samo u C
auto a = 10; // C++, a je int
\end{lstlisting}
\end{itemize}

\item Odloženo navođenje tipa funkcije\\
\inlinecode{auto ime\_funkcije(parametri) -> tip}
\begin{itemize}
    \item Koristi se kod \textit{template-ova}
    \item C++14 \inlinecode{tip} može da se izostavi
    \begin{enumerate}
        \item U tom slučaju tip se odredi preko \inlinecode{return} tipa ili definicije
        \item Funkcija ne sme da se poziva pre navođenja definicija na mestima gde je tip bitan
    \end{enumerate}
    \begin{lstlisting}
auto func(int x) -> double { ... } // return tip je double
auto f() { return 1; }
auto g();
auto a = g(); // ERROR
auto g() { return 0.5}
auto b = g() // OK
    \end{lstlisting}
\end{itemize}
\item Konstante
\begin{itemize}
    \item Izvedeni tip\\
    \inlinecode{const tip ime = vrednost;}
    \item Mora da se inicijalizuje pri definisanju
    \item Izraz ne morada bude konstantan
    \item \boxedimportant[BITNO]{Konstante inicijalizovane \textbf{konstantnim} izrazom
    (\textit{simboličke} ili \textit{kompilacione} konstante) mogu da se koriste u izrazima koji moraju biti konstantni (računaju se \textbf{u toku} prevođenja)\\
    npr. Dimenzija statičkog niza}
    
    \item \textbf{Simboličke }konstante NE alociraju memoriju
    \\\\
    \inlinecode{const char* pk = niz;} $\longleftarrow$ pokazivač na konstantu\\
    \inlinecode{char* const kp = niz;} $\longleftarrow$ konstantni pokazivač
    \item Ubacivanje \inlinecode{const} u parametre funkcije obezbeđuje da se dati objekat ne menja
    \item Ubacivanje \inlinecode{const} u \inlinecode{return} tipu funkcije obezbeđuje da se privremeni objekat rezultata ne može menjati
    \item POGLEDATI \inlinecode{constexpr}
    \begin{lstlisting}
char niz[] = { 'i', 'd', 'e', ' ', 'g', 'a', 's', '\0' };
const char* pk = niz; // Pokazivac na konstantu
pk[3] = '-';            // ERROR
pk = "OOP:(";          // OK
char* const pk = niz    // Konstantni pokazivac
pk[3] = '-';            // OK
pk = "OOP:(";          // ERROR
    \end{lstlisting}
\end{itemize}
\item Znakovne konstante
\begin{itemize}
    \item U C $\rightarrow$ \inlinecode int (\inlinecode 65 i \inlinecode 'A' su ista stvar)
    \item U C++ $\rightarrow$ \inlinecode char
    \item U izrazima \inlinecode{false} $\rightarrow$ 0, a u dodeli vrednosti logičkim promenljivama 0 $\rightarrow$ \inlinecode{false}
\end{itemize}
\newpage

\item Prostori imena (\inlinecode{namespace})
\begin{itemize}
    \item Mehanizam za izbegavanje konflikata imena\\
    \inlinecode{namespace ID \{ sadržaj \}}
    \item Jednoznačno ime
    \begin{enumerate}
        \item Celo ime \inlinecode{A::i}
        \item Uvoz imena \inlinecode{using A::i}
        \item Uvoz svih imena \inlinecode{using namespace A}
    \end{enumerate}
\end{itemize}
\item Stringovi
\begin{itemize}
    \item C stil - niz znakova koji se završava sa \inlinecode{$\backslash$0}
    \item Literal C++ stringa je \inlinecode{const char*}
    \item Literal C stringa je \inlinecode{char*}
    \item C++ - podrazumevani string je \inlinecode{""}
\end{itemize}
\item Tipovi \inlinecode{enum}, \inlinecode{struct} i \inlinecode{union}
\begin{itemize}
    \item Identifikatori ova 3 tipa mogu da se koriste kao oznaka tipa, bez ključne reči
    \item Ako u dosegu postoji objekat sa istim identifikatorom, sam ID označava objekat, a ne tip
    \begin{lstlisting}
enum RadniDan {Pon, Uto, Sre, Cet, Pet};
RadniDan r_dan = Uto;
int RadniDan;
enum RadniDan r1 = Sre; // OK
RadniDan r1 = Pon;        // ERROR
    \end{lstlisting}
\end{itemize}


\item Tip nabrajanja (\inlinecode{enum})
\begin{itemize}
    \item Svaki \inlinecode{enum} je poseban celobrojni tip
    \item Definisana je samo operacija dodele vrednosti
    \begin{enumerate}
        \item \boxedimportant[BITNO]{Eksplicitna konverzija celobrojne vrednosti u tip nabrajanja je obavezna}
        \item Ne otkriva se greška ako konvertovana vrednost nije u opsegu 
    \end{enumerate}
    \item U aritmetičkim i relacijskim izrazima, kao i pri dodeli promenljivoj tipa int, konverzija je automatska
    \begin{lstlisting}
enum Dani {PO=1, UT, SR, CE, PE, SU, NE, POSLEDNJI=7}; // NE i POSLEDNJI su 7
Dani dan=SR; // OK
Dani d=4; // ERROR - nije eksplicitna konverzija
dan++; // ERROR - nije definisana operacija ++
dan=(Dani)(dan+1); // OK
if (dan<NE) { ... } // OK
dan=(Dani)8; // Ne prijavljuje se logicka greska
    \end{lstlisting}
\end{itemize}
\item Pripadajući tip nabrajanja
\begin{itemize}
    \item Numerička reprezentacija nabrajanja
    \item Kompaktnije, podrazumeva se int\\
    \inlinecode{enum ime: pripadajući\_tip \{imenovane\_konstante\}}
    \item Paziti opsege
\end{itemize}
\begin{multicols}{2}
\item Nabrajanja sa ograničenim dosegom
\begin{itemize}
    \item Isti doseg kao i tip nabrajanja
    \item Rešenje - \inlinecode{struct} ili \inlinecode{class} iza \inlinecode{enum} 
    \item Pristup konstanti sa \inlinecode{::}
    \item Obavezna eksplicitna konverzija u ceo broj\\
    \inlinecode{int i = (int)tip::ime}
\end{itemize}
\columnbreak
\begin{lstlisting}
enum SemaforPesaci {CRVENO, ZELENO};
enum SemaforVozila {ZELENO, ZUTO, CRVENO}; // ERROR
enum struct SemaforPesaci {CRVENO, ZELENO};
enum struct SemaforVozila {ZELENO, ZUTO, CRVENO};
SemaforPesaci sp = SemaforPesaci::CRVENO;
SemaforVozila sv = ZUTO; // ERROR
int i = (int) SemaforVozila::ZELENO; // Obavezna konverzija
\end{lstlisting}
\end{multicols}
\item Inicijalizatorske liste\\
\inlinecode{\{vrednost, vrednost, ..., vrednost\}}
\begin{itemize}
    \item Inicijalizacija \textbf{svih} vrsta podataka, čak i prostih
    \item Paziti na nebezbedne konverzije
    \item Vrednosti se dodeljuju redom (čak i strukturama, uniji se popuni prvo polje)
    \item Manjak vrednosti $\rightarrow$ popunjava se nulama
    \item Višak vrednosti $\rightarrow$ Greška
    \item Argumenti funkcija i izrazi u \inlinecode{return} mogu biti ove liste
    \item Bezimeni podatak
    \item \boxedimportant[ČUDNO]{Niz ne može da dobije vrednost liste nakon inicijalizacije, sem ako je deo neke strukture}
    \begin{lstlisting}
int i1={1}, i2{1}, i3={i1+i2};
i1={2};
int i4={0.5}; // ERROR - nije bezbedno
int *pi={&i1};
int n1[5]={1,2,3}, n2[5]{1,2,3}, n3[]{1,2,3};
int m[][3]{{1,2},{},{1,2,3}};
n1={4,5,6}; // ERROR
struct S1{int a,b;};
S1 s11={1,2}, s12{1,2}; s11={3,4};
struct S2{int a; S1 b; int c[3];};
S2 s21={1,{2,3}, {4,5,6}}, s22{1,2,3,4,5,6};
s21 = {6, {5,4}, {3,2,1}}; 
    \end{lstlisting}
\end{itemize}
\item Bezimena unija
\begin{itemize}
    \item Predstavlja objekat koji sadrži u raznim trenucima razne tipove podataka
    \item Datotečki ili blokovski doseg
    \item Unija za koju je definisan barem 1 objekat ili pokazivač $\rightarrow$ nije bezimena unija, iako nema ime
    \begin{lstlisting}
union{ int i; double d; char *pc; };
i=55; d=123.456; pc="ABC";
    \end{lstlisting}
\end{itemize}

\item Uvek promenljiva polja (\inlinecode{mutable})
\begin{itemize}
    \item Polje označeno sa \inlinecode {mutable} može da se menja čak i za \inlinecode {const} parametre
    \begin{lstlisting}
struct X{
    int a;
    mutable int b;
};
int main(){
    X x1;
    const X x2;
    x1.a = 4;
    x1.b = 2;
    x2.a = 3; // ERROR
    x2/b = 4; // OK
}
    \end{lstlisting}
\end{itemize}

\newpage
\item Dinamički objekti
\begin{itemize}
    \item \inlinecode {new/delete}
    \item Operand operatora \inlinecode {new} je identifikator tipa \inlinecode T sa eventualnim inicijalizatorima
    \begin{enumerate}
        \item Alocira potreban prostor za objekat datog tipa
        \item Poziva konstruktor tipa
    \end{enumerate}
    \item Ako nema mesta \inlinecode {bad\_alloc} exception\\
    U nestandardizovanom C++ vraća se \inlinecode {nullptr}
    \item Vraća pokazivač na kreirani objekat\\
    \inlinecode{T *t = new (nothrow)T;} $\longleftarrow$ Ignorisanje exceptiona, vraća \inlinecode{nullptr} ako ne uspe
    \item Stavlja na \textit{heap}
\end{itemize}
\item Uništavanje dinamičkih objekata
\begin{itemize}
    \item \inlinecode {delete} ima 1 operand (pokazivač nekog tipa)
    \item Mora biti objekat kreiran pomoću \inlinecode {new}, inače će ponašanje biti nepredviđeno
    \item  \inlinecode{delete nullptr} ne radi ništa
    \begin{enumerate}
        \item Poziva destruktor za pokazani objekat
        \item Oslobađa zauzeti prostor
    \end{enumerate}
    \item \inlinecode {delete} vraća \inlinecode {void}
\end{itemize}
\item Dinamički nizovi
\begin{itemize}
    \item Sve dimenzije niza osim prve moraju biti konstanti izrazi
    \item Inicijalizacija
    \begin{enumerate}
        \item Podrazumevani konstruktor ILI
        \item Generisain konstruktor
    \end{enumerate}
    \inlinecode{delete [] pt;}
    \item Redosled konstrukcije po rastućem indeksu
    \item Redosled destrukcije obrnut od redosleda konstrukcije
    \item Može inicijalizatorska lista
\end{itemize}


\item Reference
\begin{itemize}
    \item C isključivo po vrednosti prenosi argumente
    \item C++ prenosi argumente i po referenci
    \begin{lstlisting}
void f(int i, int &j){ // i po vrednosti, j po referenci
    i++; // stvarni argument se nece promeniti
    j++; // stvarni argument ce se promeniti
}
int main () {
    int si=0,sj=0;
    f(si,sj);
    cout<<"si="<<si<<", sj="<<sj<<endl;
}
Izlaz: si=0, sj=1
    \end{lstlisting}
\end{itemize}
\item Definisanje referenci
\begin{itemize}
    \item Reference na LVrednosti (\inlinecode {lvalue})
    \item Znak \inlinecode\& ispred imena
    \item Sinonim za objekat, ne može se promeniti
    \item U definiciji mora da se inicijalizuje objektom
    \item Svaka naredba nad referencom je operacija nad pokazanim objektom
    \begin{lstlisting}
int i=1; // celobrojni objekat i
int &j=i; // j upucuje na i
i=3; // menja se i
j=5; // opet se menja i
int *p=&j; // isto sto i &i
j+=1; // isto sto i i+=1
int k=j; // posredan pristup do i preko reference
int m=*p; // posredan pristup do i preko pokazivaca
    \end{lstlisting}
\end{itemize}


\item Implementacija referenci
\begin{itemize}
    \item Slično konstanom pokazivaču\\
    \inlinecode{int \&j = *new int(2);}\\
    \inlinecode{delete \&j;}
    \item Ako je referenca na \inlinecode {const} objekat, ne sme se menjati 
    \item Ne postoje nizovi referenci, pokazivači na referencu, kao ni reference na reference
    \item Referenca na pokazivač je dozvoljena
\end{itemize}


\item Funkcije koje vraćaju reference
\begin{itemize}
    \item Funkcija mora da vrati referencu na objekat koji je \textit{živ} i posle funkcije
    \item Rezultat poziva funkcija je LVrednost (\inlinecode {lvalue}) samo kao funkcija vraća referencu
    \begin{lstlisting}
int& f(int &i) { int &r = *new int(i); return r; } // OK
int& f(int &i) { return *new int(i); } // OK
int& f(int &i) { return i; } // OK
int& f(int &i) { int r = i; return r;} // NIJE OK
int& f(int i) { return i; } // NIJE OK
int& f(int &i) { int r = *new int(i); return r; } // NIJE OK
int& f(int &i) { int j = i; &r = j; return r; } // NIJE OK
    \end{lstlisting}
\end{itemize}
\item Obilazak elemenata niza u petlji\\
\inlinecode{for(tip ime: niz) naredba}
\begin{itemize}
\item \textbf{foreach}, range petlja
\item Može se staviti referenca na objekat, čime omogućavamo menjanje svakog elementa kom pristupamo - bez njega su \textit{read-only}
\begin{lstlisting}
for(auto& it: arr){ // it od iterator
    cout<<it++<<endl;
}
\end{lstlisting}

\end{itemize}
\begin{multicols}{2}
\item Reference na DVrednosti (\inlinecode {rvalue})
\begin{itemize}
    \item Tip reference na DVrednost\\
    \inlinecode{osnovni\_tip \&\& ime = vrednost;}
    \item Referenca na DVrednost je LVrednost
    \item Posledica - privremeni podaci dobijaju imena, pa možemo da ih menjamo
    \item Podatak može biti nepromenljiv
\end{itemize}
\columnbreak
\begin{lstlisting}
int i=1; // i je promenljiv podatak
const int ci=i; // ci je nepromenljiv podatak
int && rd1=i; // ERROR - i je promenljiva lvrednost
int && rd2=ci; // ERROR - ci je nepromenljiva lvrednost
int && rd3=i+1; // (i+1) je promenljiva dvrednost
int && rd4=10; // 10 je nepromenljiva dvrednost 
rd3++; rd4++; // rd3==3, rd4==11 
\end{lstlisting}
\end{multicols}
\item Reference na DVrednosti kao parametri
\begin{itemize}
    \item Ne postoji \textit{bočni efekat}
    \item \inlinecode {const} nema smisla
\end{itemize}
\item Podrazumevane vrednosti argumenata
\begin{itemize}
    \item Može biti samo nekoliko poslednjih argumenata
    \item Proizvoljni izrazi
\end{itemize}
\newpage

\item Neposredno ugrađivanje funkcije u kod
\begin{itemize}
    \item Jednostavne, kratke funkcije\\
    \inlinecode{inline tip ime( parametri ) \{ definicija }\}
    \item Izbegavanje prenosa argumenata i poziva funkcija
    \item Funkcija članica klase je \inlinecode {inline} ako se definiše u definiciji klase
    \item Ako se definiše van definicije klase, mora se staviti ključna reč
    \item Prevodilac ne mora da poštuje \inlinecode {inline}
    \item Ako je u više datoteka, mora se definisati u svakoj
    \item Često rešenje je sprovođenje sa dodatnom datotekom-zaglavljem, ali se tad funkcija može direktno videti od strane drugih korisnika
    \item Eliminiše potrebu za makroima
    \begin{lstlisting}
#define max(i,j)((i)>(j))?(i):(j)
max(i++, k++);
((i++)>(j++))?(i++):(j++); // 2x se inkrementira
    \end{lstlisting}
\end{itemize}

\item Preklapanje imena funkcija
\begin{itemize}
    \item \textit{Function name overloading}
    \item Funkcije koje realizuju logički istu operaciju, sa različitim tipovima argumenata
    \item U C nema preklapanja - funkcije moraju imati različita imena
    \item Mora da se razlikuje broj i/ili tip argumenata
    \item Tip rezultata \textbf{ne mora} da se razlikuje
    \item Takođe, \textbf{nije dovoljno} da se razlikuje samo \inlinecode{return} tip
    \item Statički koncept, sve se odvija u prevođenju
    \item Prevodilac prioritira slaganje tipova
    \begin{enumerate}
        \item Potpuno slaganje - uključuje niz $\rightarrow$ pokazivač, ili referenca $\rightarrow$ objekat
        \item Slaganje standardnim konverzijama\\
        npr. \inlinecode{char} $\rightarrow$ \inlinecode{int}
        \item Slaganje korisničkim konverzijama
    \end{enumerate}
    \begin{lstlisting}
double max (double i, double j)
{ return (i>j) ? i : j; }
const char* max (const char *p, const char *q)
{ return (strcmp(p,q)>=0)?p:q; }
double r=max(1.5,2.5); // max(double,double)
double p=max(1,2.5); // (double)1; max(double,double)
const char *q=max("Pera","Mika");// max(const char*,const char*)
    \end{lstlisting}
\end{itemize}

\item Pristup elementima
\begin{itemize}
    \item Složeni podaci
    \item Problem 2 definicije koje imaju identično telo sa različitim parametrima
    \item Druga funkcija poziva prvu $\longleftarrow$ Rešenje
    \item Slično za pokazivače i reference
    \item Čudan slajd, izgleda kao dodatno objašnjavanje overloadinga
    \begin{lstlisting}
int& elem( int *a, int i) { return a[i]; }
const int& elem(const int *a, int i) { return a[i]; }
int a[20],i=10;
const int b[20]={0};
elem(a,i)=1;
elem(b,i)=1; // ERROR
int x=elem(b,i);
    \end{lstlisting}
\end{itemize}
\newpage
\item Napomene 
\begin{itemize}
    \item Uputstva za prevodioca, \textbf{anotacije}\\
    \inlinecode{[[napomena]]}
    \item Služe prevodiocu za provere i optimizacije
    \item Prevodilac može da ih zanemari
\end{itemize}
\item Funkcije koje se ne vraćaju
\begin{itemize}
    \item Postoje funkcije koje se ne vraćaju na mesto poziva
    \item Nasilno prekida rad programa sa \inlinecode{exit(kod)}
    \begin{enumerate}
            \item Kod = 0 \checkmark{}
            \item Kod $\neq$ 0 X
    \end{enumerate}

    \item Anotacija \inlinecode{[[noreturn]]}
    \item Ako ima negde \inlinecode{return}, kod postaje nepredvidiv
\end{itemize}
\item Operatori i izrazi
\begin{itemize}
    \item Novi operatori (12)
    \begin{itemize}
        \item[] unarni \inlinecode{::, ::, new, delete, .*, ->*, typeid, throw, alignof}, 4 \textit{cast} operatora
    \end{itemize}
    \item Postfiksni \inlinecode{++, --} imaju viši prioritet od prefiksnih
    \item Prefiksni \inlinecode{++, --} $\rightarrow$ \inlinecode{lvalue}
    \item Dodela vrednosti $\rightarrow$ \inlinecode{lvalue}
    \item Ternarni operator je \inlinecode{lvalue} ako su drugi i treći operator \inlinecode{lvalue}
\end{itemize}


\item Operatori konverzije tipa
\begin{itemize}
    \item C \textit{cast} \\
    \inlinecode{(tip) izraz} $\longleftarrow$ Ne preporučuje se
    \item Novi \textit{cast} operatori
    \begin{enumerate}
        \item \inlinecode{static\_cast <oznaka\_tipa> (izraz)}
        \item \inlinecode{reinterpret\_cast <oznaka\_tipa> (izraz)}
        \item \inlinecode{const\_cast <oznaka\_tipa> (izraz)}
        \item \inlinecode{dynamic\_cast <tip\_pokazivača\_ili\_reference> (izraz)}
    \end{enumerate}
    \item Bezbedne i nebezbedne konverzije\\
    \inlinecode{int} $\rightarrow$ \inlinecode{float} \checkmark{}\\
    \inlinecode{float} $\rightarrow$ \inlinecode{int} X
    \item Notacija je nezgrapna i kabasta iz 2 razloga
    \begin{enumerate}
        \item Lakše se uoči u tekstu
        \item Da programeri ne bi koristili
    \end{enumerate}
    \item Ako postoji potreba za eksplicitnim konverzijama $\rightarrow$ Preispitati projektne odluke
\end{itemize}
\item Statička konverzija
\begin{itemize}
    \item Prenosive konverzije
    \begin{enumerate}
        \item Između numeričkih tipova
        \item Između pokazivača i \inlinecode{void*}
        \item Nestandardne konverzije - definiše programer
    \end{enumerate}
    \item Primenjuju se automatski kad su bezbedne
    \item Eksplicitan poziv kad je nebezbedno\\
    npr. \inlinecode{void*} $\rightarrow$ drugi pokazivač, numerički tip $\rightarrow$ \inlinecode{char}
    \item \inlinecode{nullptr} može da se dodeli bilo kom tipu pokazivača
    \item Ne preporučuje se korišćenje \inlinecode{NULL} ili \inlinecode0
\end{itemize}
\item Reinterpretirajuća konverzija
\begin{itemize}
    \item Konverzija tipova bez logičke veze\\
    npr. \inlinecode{int} $\rightarrow$ pokazivač
    \item Nema pretvaranja vrednosti, istu vrednost različito interpretiramo
    \item Jako nebezbedno
\end{itemize}
\item Konstanta konverzija
\begin{itemize}
    \item Dodavanje ili uklanjanje \inlinecode{const}
    \item Dodavanje je bezbedno, uklanjanje nije
\end{itemize}
\end{xitemize}
\newpage

\section{Klase i objekti}
\begin{xitemize}
\item Osnovni pojmovi
\begin{itemize}
    \item Klasa je struktuirani korisnički tip koji obuhvata
    \begin{enumerate}
        \item Podatke koji opisuju stanje objekta klase
        \item Funkcije namenjene definisanju operacija nad podacima
    \end{enumerate}
    \item Klasa je formalni opis aptrakcije koja ima
    \begin{enumerate}
        \item Internu implementaciju
        \item Javni interfejs
    \end{enumerate}
    \item Instanca klase $\rightarrow$ objekat
    \item Podaci klase $\rightarrow$ \textbf{atributi}, polja, podaci članovi
    \item Funkcije klase $\rightarrow$ \textbf{metodi}, primitivne operacije, funkcije članice
\end{itemize}
\item Komunikacija objekata
\begin{itemize}
    \item Objekti klasa komuniciraju da ostvare složene funkcije
    \item Poziv metoda $\rightarrow$ \textbf{upućivanje poruke}
    \item Objekat može da menja stanje kad se pozove metod
    \item \textbf{Objekat-klijent} - poziva metod
    \item \textbf{Objekat-server} - metod mu je pozvan
    \item Iz svog metoda se može pozvati metod drugog objekta iste ili druge klase
    \item Unutar metode, članovima objekta-servera pristupa se navođenjem imena
\end{itemize}
\item Pravo pristupa
\begin{itemize}
    \item Sekcije
\end{itemize}
\begin{enumerate}
    \item \inlinecode{private}
    \begin{itemize}
        \item Zaštićeni od spolja (\textbf{kapsulirani})
        \item Pristupaju im samo metodi klase
    \end{itemize}
    \item \inlinecode{protected}
    \begin{itemize}
        \item Dostupni metodima iste klase + sve klase izvedene iz nje
    \end{itemize}
    \item \inlinecode{public}
    \begin{itemize}
        \item Dostupni spolja bez ograničenja
    \end{itemize}
\end{enumerate}
\begin{itemize}
    \item Preporučuje se redosled \inlinecode{private} $\rightarrow$ \inlinecode{protected} $\rightarrow$ \inlinecode{public}
    \item Može da postoji više sekcija iste vrste
    \item Podrazumevana labela je \inlinecode{private}

\item \boxedimportant[BITNO]{Kontrola pristupa je stvar klase, a ne objekta\\
Metod jednog objekta može da pristupa privatnim članovima drugog objekta iste klase}
\item Kontrola pristupa je odvojena od koncepta dosega
\begin{enumerate}
    \item Odredi se postojanje
    \item Proveravanje prava pristupa
\end{enumerate}
\end{itemize}
\newpage
\item Definisanje klase
\begin{itemize}
    \item Atributi
    \begin{enumerate}
        \item Mogu da budu i inicijalizovane - od C++11
        \item Ne moguda budu tipa klase koja se definiše, ali su dozvoljeni pokazivači i reference na tu klasu
    \end{enumerate}
    \item Metodi
    \begin{enumerate}
        \item U definiciji mogu da se 
        \begin{itemize}
            \item[-] Deklarišu - samo prototip
            \item[-] Definišu - kompletno telo
        \end{itemize}
        \item Funkcije definisane u definiciji klase su \inlinecode{inline} i mogu pristupati članovima imenom
        \item Funkcije koje su samo deklarisane u definiciji klase moraju biti definisane kasnije, van definicije, sa proširenim dosegom za pristup članovima\\
        \inlinecode{<ime\_klase>::<ime\_funkcije>}
        \item Vrednost rezultata metoda može biti tipa klase koja se definiše, kao i pokazivač ili referenca na nju
    \end{enumerate}
    \item Definicija se piše tamo gde se klasa koristi, obično u \textit{header} fajl (\inlinecode{.h})
    \item Nepotpuna definicija klase je \textbf{deklaracija}
    \item Pre defincije, a posle deklaracije
    \begin{enumerate}
        \item Mogu da se definišu pokazivači i reference
        \item Ne mogu da se definišu objekti te klase
    \end{enumerate}
\end{itemize}
\item Objekti klase
\begin{itemize}
    \item Uobilajeno definisanje, kao kod standardnih tipova
    \item Za svaki objekat formira se poseban komplet svih nestatičkih atributa
    \item Nestatički metodi se pozivaju za objekte, a statički za klase
    \item  Lokalne \inlinecode{static} promenljive metoda
    \begin{enumerate}
        \item Zajedničke za sve objekte
        \item Žive od nailaska na njih do kraja programa
        \item \boxedimportant[WTF]{Imaju svojstva lokalnih promenljivih globalnih funkcija}
    \end{enumerate}
\end{itemize}
\item Podrazumevane operacije
\begin{itemize}
    \item Definisanje objekata, pokazivača i referenci na objekte i nizove objekata
    \item Dodela vrednosti jednog objekta drugo
    \item Uzimanje adrese \inlinecode\&
    \item Pristupanje objektu preko pokazivača \inlinecode*
    \item Pristupanje atributima i pozivanje metode neposredno pomoću \inlinecode.
    \item Pristupanje atributima i pozivanje metoda posredno pomoću pokazivača \inlinecode{->}
    \item Pristupanje elementima niza \inlinecode{[]}
    \item Prenošenenje objekta kao argumenata po vrednosti, referenci ili pokazivaču
    \item Vraćanje objekta iz funkcije po vrednosti, referenci ili pokazivaču
    \item Preklapanje operatora može redefinisati dosta gorenavedenog
\end{itemize}


\item Pokazivač \inlinecode{this}
\begin{itemize}
    \item Pokazivač na tekući objekat
    \item Unutar svako nestatičkog objekta je implicitno, \inlinecode{this} je skriveni argument svakog metoda\\
    \inlinecode{objekat.f()} $\sim$ \inlinecode{f(\&objekat)}
     \item Konstanti pokazivač na klasu čiji je metod član\\
    Klasa \inlinecode{X}, \inlinecode{this} $\rightarrow$ \inlinecode{X* const}
    \item Pristup se obavlja neposredno
    \item Primeri korišćenja
    \begin{enumerate}
        \item Tekući objekat vratiti kao rezultat metoda
        \item Adresa objekta je potrebna kao argument
        \item Tekući objekat ubaciti u listu
    \end{enumerate}
    \begin{lstlisting}
// Definicija metoda zbir(Kompleksni) klse Kompleksni
Kompleksni Kompleksni::zbir(Kompleksni C){
    Kompleksni t = *this; // u t se kopira tekuci objekat
    t.real+=c.real;
    t.imag+=c.imag;
    return t;
}
//...
int main(){
    Kompleksni c, c1,c2;
    //...
    c=c1.zbir(c2);
}
    \end{lstlisting}
\end{itemize}
\item Inspektori i mutatori
\begin{itemize}
    \item \textbf{Inspektor} ili selektor $\rightarrow$ ne menja stanje objekta
    \item \textbf{Mutator} ili modifikator $\rightarrow$ menja stanje objekta
    \item Dobra praksa da se kaže koji tip od ova dve je metod
    \item \inlinecode{const} iza liste parametara $\rightarrow$ inspektor
    \item Postoji konstantan metod, ali je to druga stvar
\end{itemize}

\item Definisanje inspektora\\
\inlinecode{<tip> ime(parametri) const \{definicija\}}
\begin{itemize}
    \item Notaciona pogodnost
    \item Prevodilac nema načina da osigura da inspektor ne menja atribute\\ Eksplicitna konverzija može da probiju konrolu konstantnosti
    \item U inspektoru \inlinecode{this} je \inlinecode{const X* const} 
    \item Nije moguće menjati objekat pomoću \inlinecode{this}
    \item Za nepromenljive objekte nije dozvoljeno pozivati metod koji nije inspektor
    \begin{lstlisting}
class X {
public:
    int citaj () const { return i; }
    int pisi (int j=0) { int t=i; i=j; return t; }
private:
    int i;
};
X x; const X cx;
x.citaj(); // OK - inspektor promenljivog objekta
x.pisi(); // OK - mutator promenljivog objekta
cx.citaj(); // OK - inspektor nepromenljivog objekta
cx.pisi(); // ERROR - mutator nepromenljivog objekta
    \end{lstlisting}
\end{itemize}
\newpage
\item Nepostojani metodi (\inlinecode{volatile})
\begin{itemize}
    \item Suprotnost konstantnog metoda
    \item Veza sa konkurentnim programiranjem
    \item Neki drugi \textit{thread} može u svakom trenutku da promeni stanje objekta
    \item Prevodilac ne izvršava optimizaciju
    \item \inlinecode{volatile} može da se poziva za nepostojane i promenljive objekte
    \item \inlinecode{const volatile} - za sve vrste objekata
    \begin{lstlisting}
class X {
public:
    X(){ kraj=false; }
    int f() volatile { // da nije volatile, moguca optimizacija:
    while(!kraj){/*...*/} // if (!kraj) while() {/*...*/}
    } // u telu (...) se ne menja kraj
    void zavrseno(){ kraj=true; }
private:
    bool kraj;
    \end{lstlisting}
\end{itemize}

\item Modifikatori metoda \inlinecode{\&} i \inlinecode{\&\&}
\begin{itemize}
    \item Bez modifikatora \inlinecode{\&} i \inlinecode{\&\&} metod se može primeniti na \inlinecode{lvalue} i \inlinecode{rvalue}
    \item Modifikator \inlinecode{\&} - tekući objekat može biti samo \inlinecode{lvalue}
    \item Modifikator \inlinecode{\&\&} - tekući objekat može biti samo \inlinecode{rvalue}
    \item \boxedimportant[BITNO]{ Mogu da postoje metodi čiji se potpisi razlikuju samo po ovom modifikatoru}
    \begin{lstlisting}
class U {
public:
    int f() & {return 1;}
    int f() const & {return 2;}
    int f() && {return 3;}
};
U u1; const U u2=u1;
int i = u1.f(); int j = u2.f(); int k = U().f();
    \end{lstlisting}
\end{itemize}

\item Pojam konstruktora
\begin{itemize}
    \item Specifična funkcija klase koju definiše početno stanje objekta
    \item Isto ime kao klasa
    \item Nema \inlinecode{return} tip, čak ni \inlinecode{void}
    \item Proizvoljan broj proizvoljnih tipova parametara
    \begin{enumerate}
        \item Ne sme biti tip klase koju definiše ako je jedini parametar ili ako svi ostali imaju podrazumevanu vrednost
        \item Dozboljen tip pokazivača na \inlinecode{lvalue} i \inlinecode{rvalue} date klase
    \end{enumerate}
    \item Implicitno se poziva prilikom kreiranja
    \item Pristup članovima objekta kao i bilo koji drugi metod
    \item Može biti preklopljen $\backslash$ \textit{overloaded}
\end{itemize}
\item Podrazumevani konstruktor
\begin{itemize}
    \item Može se pozvati bez stvarnih argumenata - nema parametre ili su svi podrazumevani
    \item Ugrađeni podrazumevani konstruktor je bez parametara i ima prazno telo
    \item Ugrađeni konstruktor postoji smao ako klasa nije definisala nijedan drugi konstruktor
    \item Definisanje nekog konstruktora se suspenduje ugrađeni\\
    Restauracija ugrađenog konstruktora - deklaracija iza koje sledi \inlinecode{= default}
    \item Kad se kreira niz objekata poziva se podrazumevani konstruktor po rastućem redosledu indeksa
\end{itemize}
\newpage
\item Pozivanje konstruktora
\begin{itemize}
    \item Stvaranje bilo kakvog objekta
\end{itemize}
\begin{enumerate}
    \item Definicija statičkog objekta
    \item Definicija automatskog objekta
    \item Dinamički objekat kreiran operatorom \inlinecode{new}
    \item Kad se stvarni argument klasnog tipa prenosi u formalni
    \item Kada se kreira privremeni objekat pri povratku iz funkcije
\end{enumerate}
\item Argumenti konstruktora
\begin{itemize}
    \item Pri stvaranju objekta moguće je navesti inicijalizator iza imena
    \item Inicijalizator sadrži listu argumenata konstruktora u zagradama
    \begin{enumerate}
        \item \inlinecode{()} ili \inlinecode{\{\}}
        \item Ako \inlinecode{\{\}} $\rightarrow$ može se pisati i \inlinecode{= \{...\}}
        \item Nisu dozboljene prazne zagrade \inlinecode{()}\\
        Deklaracija funkcije
    \end{enumerate}
    \item Moguća notacija \inlinecode{<objekat>  = <vrednost>}
    \item Poziva se onaj konstruktor koji se najbolje salže po potpisu
    \item Može da ima podrazumevane vrednosti
    
    \begin{multicols}{2}
    \begin{lstlisting}
class X {
    char a; int b;
public:
    X ();
    X (char, int=0);
    X (const char*);
    X(X); // ERROR
    X(X*);
    X(X&);
    X(X&&);
};
X f () {
    X x1; // X()
    X x2{}; // X()
    X x3={};// X()
    X x(); // dekl. f-je
    return x1;
}
\end{lstlisting}
\columnbreak
\begin{lstlisting}
void g () {
    char c='a';
    const char *p="Ne volim OOP";
    
    X x1(c); // X(char,int)
    X x2=c;
    X x3(c,10);
    X x4{c,20};
    X x5={c,30};
    X x6(p); // X(char*)
    X x7(x1); // X(X&)
    X x8{x1};
    X x9={x1};
    X x10=f(); // X(X&&)
    X* p1=new X;// X()
    X* p2=new X(); // X(char,int)
    X* p4=new X{c,10};
}
    
    \end{lstlisting}
    \end{multicols}
\end{itemize}
\newpage
\item Konstrukcija članova
\begin{itemize}
    \item Pre izvršavanja tela konstruktora
    \begin{enumerate}
        \item Inicijalizuju se prosti tipovi
        \item Pozivaju se konstruktori za klasne tipove
    \end{enumerate}
    \item Inicijalizatori mogu da se navedu u zaglavlju definicije (NE deklaracije) konstruktora, iza znaka \inlinecode:
    \item \boxedimportant[]{Ako atributi ima inicijalizatoru telu klase i u definiciji konstruktora $\rightarrow$ primenjuje se inicijalizator iz definicije konstruktora}
    \item \boxedimportant[BITNO]{Inicijalizacija atributa - \textbf{redosled navođenja u definiciji klase}
    \begin{itemize}
        \item[-] Bez obzira da li su primitivni ili klasni tipovi
        \item[-] Bez obzira na redosled u listi inicijalizatora
    \end{itemize}}\\\\
    \inlinecode{class X \{\\
    private:}\\
    \hspace*{0.5cm}
    \inlinecode{int i = 0; \\
    \}
    }
    \item Do C++11 nije bila dozvoljena inicijalizacija atributa u definiciji klase
    \item Inicijalizacija je različita od operacije dodele koja se može vršiti jedino unutar tela konstruktora
    \item Inicijalizacija je neophodna
    \begin{enumerate}
        \item Kada ne postoji podrazumevani konstruktor klase atributa
        \item Kada je atribut nepromenljiv
        \item Kada je atribut referenca
    \end{enumerate}
    \begin{lstlisting}
class YY { public: YY (int j) {...} };
class XX {
    YY y; int i=0;
public:
    XX (int);
};
XX::XX (int k) : y(k+1), i(k-1) {...} // y=k+1, i=k-1
    \end{lstlisting}
    \begin{lstlisting}
// Primer konstrukcija dva objekta od kojih jedan sadrzi drugi
class Kontejner {
public:
    Kontejner () : deo(this) {...}
private:
    Deo deo;
};
class Deo{
public:
    Deo(Kontejner* kontejner):mojKontejner(kontejner) {...}
private:
    Kontejner* mojKontejner;
};
    \end{lstlisting}
    \end{itemize}
\begin{multicols}{2}
\item Delegirajući konstruktor
\begin{itemize}
    \item U listi inicijalizatora definicije delegiruajućeg konstruktora može da se navede poziv drugog konstruktora
    \item Pre izvršenja tela delegirajućeg konstruktora, izvršava se ciljni konstruktor
\end{itemize}
\columnbreak
\begin{lstlisting}
class T {
    T(int i){}
    T():T(1){} // delegirajuci: T(), ciljni: T(int)
    T(char c): T(0.5){} // ERROR - rekurzija
    T(double d): T('a'){}   
}
\end{lstlisting}
\end{multicols}
\begin{itemize}
     \item Kad se navodi ciljni konstruktor, navodi se samo on
    \item Ako dolazi do neposrednog ili posrednog delegiranja $\rightarrow$ greška\\
    Prevodilac ne otkriva ovakav tip greške
\end{itemize}
\item Eksplicitni poziv konstruktora
\begin{itemize}
    \item Ovakav poziv kreira primvremeni objekat klase pozivom odgovarajućeg konstruktora
    \item Isto se dešava ako se u inicijalizatoru objekta eksplicitno navede poziv konstruktora\\
    \inlinecode{Kompleksni c = Kompleksni(0.1, 5);}\\
    Privremeni objekat se kopira u \inlinecode c - zavisi od prevodioca
\end{itemize}
\item Konstruktor kopije
\begin{itemize}
    \item Kopirajući konstruktor
    \item Pri inicijalizaciji objekta O1 drugim objektom O2 iste klase poziva se konstruktor kopije
    \item Ugrađeni, implicitno definisani, konstruktor kopije
    \begin{enumerate}
        \item Vrši inicijalizaicju članova O1 članovima O2 (pravi \textbf{plitku kopiju} - \textit{shallow copy})
        \item Primitivni atributi se prosto kopiraju - uključujući i pokazivače
        \item Za klasne atribute se pozivaju njihovi konstruktori kopije
    \end{enumerate}
    \item Ugrađeni konstruktor kopije se briše ili suspenduje
    \begin{enumerate}
        \item Eksplicitno\\
        \inlinecode{X(const X\&) = delete}
        \item Implicitno - pisanjem premeštajućeg konstruktora ili premeštajućeg operatora deodele\\
        Restauriranje konstruktora kopije \inlinecode{X(const X\&) = default}
    \end{enumerate}
    \item Problem pokazivača $\rightarrow$ pravimo \textbf{duboku} kopiju - \textit{deep copy}
    \item Parametri konstruktora kopije su \inlinecode{X\&} ili \inlinecode{const X\&}
    \item Ostali eventualni parametri moraju biti podrazumevane vrednosti
\end{itemize}
\item Pozivanje konstruktora kopije
\begin{itemize}
    \item Poziva se jednim stvarnim argumentom
    \item Konstruktor kopij se poziva kada se objekat inicializuje objektom iste klase i to
    \begin{enumerate}
        \item Prilikom stvaranja trajnog, automatskog, dinamičkog ili privremenog objekta
        \item Prilikom prenosa argumenata po vrednosti u funkciju (stvara se automatski objekat)
        \item Prilikom vraćanja vrednosti iz funkcije (stvara se privremeni objekat
    \end{enumerate}
    \item Prevodilac sme da preskoči poziv konstruktora kopije zbog optimizacije
    \begin{itemize}
        \item[-] Ako se stvarani objekat inicijalizuje privremenim objektom iste klase
        \item[-] Izostaju bočno efekti koje programer očekuje
        \item[-] Čak i tada mora postojati konstruktor kopije ili premeštajući konstruktor
    \end{itemize}
    \begin{lstlisting}
class XX {
public:
    XX (int);
    XX (const XX&); // konstruktor kopije
    //...
};
XX f(XX x1) {
    XX x2=x1; // poziv konst. kopije XX(XX&) za x2
    return x2; // poziv konst. kopije za privremeni
} // objekat u koji se smesta rezultat
void g() {
    XX xa=3, xb=1;
    xa=f(xb); // poziv konst. kopije samo za parametar x1,
// a u xa se samo prepisuje
// privremeni objekat rezultata, ili se
} // poziva XX::operator= ako je definisan
    \end{lstlisting}
\end{itemize}
\newpage
\item Premeštajući konstruktor
\begin{itemize}
    \item Konstruktor koji se poziva za konstrukciju objekta istog tipa, pri čemu je izvorišni objekat na kraju životnog veka
    \item Izvorišni objekat je \textbf{nvrednost} (nestajuća vrednost) - \textit{xvalue (expiring value)}
    \item Izvorišni objekat ne mora da se sačuva
    \item Samo premestimo njegove dinamičke delove u odredišni objekat
    \item Nema kopiranja dinamičkih delova
    \item Posledica $\rightarrow$ premeštajući konstruktor je efikasniji od kopirajućeg
    \item Modifikovati izvorišni objekat da njegova destrukcija ne povuče razaranje premeštenih delova
    \item Postoji ugrađeni, implicitno definisani, premeštajući konstruktor, ali ona ima problem - ne briše originalne pokazivače u izvorišnom objektu
    \item Ugrađeni premeštajući konstruktor se briše ako se eksplicitno definiše bar jedan od navedenih: \marginnote{Nisam 100\% siguran za ovo BAR}
    \begin{enumerate}
        \item Premeštajući konstruktor
        \item Kopirajući konstruktor
        \item Destruktor
        \item Operator dodele
    \end{enumerate}
    
\end{itemize}

\item Pozivanje premeštajućeg konstruktora
\begin{itemize}
    \item Paramater je \inlinecode{X\&\&}, ostali su podrazumevani parametri \marginnote{BEZ CONST}
    \item Prevodilac poziva premeštajući konstruktor
    \begin{enumerate}
        \item Ako izvorišni objekat nestaje
        \item Ako u klasi postoji premeštajući konstruktor
    \end{enumerate}
    \item Ako u klasi ne postoji premeštajući
    \begin{enumerate}
        \item Poziva se kopirajući konstruktor
        \item Semantika je ista
        \item Promena je samo u efikasnosti
    \end{enumerate}
    \begin{lstlisting}
class Niz {
    double* a; int n;
public: ... Niz( Niz&& niz ){ a=niz.a; niz.a=nullptr; n=niz.n; }
} ...
Niz f(Niz niz){ return niz; }
    \end{lstlisting}
\end{itemize}
\item Konverzioni konstruktor
\begin{itemize}
    \item Konverzija između tipova od kojih je bar jedan klasa
    \item Odredišni tip mora biti klasa\\
    \inlinecode{X::X(T\&) X:: X(T)} $\rightarrow$ konverzija tipa T u X
    \item Korisničke konverzije se primenjuju automatski ako je jednoznačan izbor konverzije, izuzev u slučaju \inlinecode{explicit} konstruktora
    \item Konverzija mora biti posredna\\
    \inlinecode{U::U(T\&), V::V(U\&) } $\rightarrow$ \inlinecode{V(U(t))} eksplicitno
    \item Nije moguće konvertovati u primitivni tip
    \item Konverzija argumenata i rezultat funkcije
    \begin{enumerate}
        \item Pri pozivu funkcije
        \begin{itemize}
            \item[-] Inicijalizuju se parametri stvarnim argumentima uz eventualnu konverziju tipa
            \item[-] Parametri se ponašaju kao automatski lokalni objekti pozvane funkcije
            \item[-] Ovi objekti se konstruišu pozivom odgovarajućih konstruktora
        \end{itemize}
        \item Pri povratku iz funkcije
        \begin{itemize}
            \item[-] Konstruiše se privremeni objekat koji prihvata vrednost \inlinecode{return} izraza na mestu poziva
        \end{itemize}
    \end{enumerate}
    \newpage
    \begin{lstlisting}
//Konverzioni konstruktor - PRIMER
class T {
public:
    T(int i); // Konstruktor
};
T f (T k) {
    //...
    return 2; // Poziva se konstruktor T(2)
}
int main () {
    T k(0);
    k=f(1); // Poziva se konstruktor T(1)
    //...
}
    \end{lstlisting}
\end{itemize}
\item Destruktor
\begin{itemize}
    \item Specifična funkcija članica koja uništava objekat
    \item Nosi isto ime kao klasa, uz $\sim$ ispred imena
    \item Nema tip rezltata i ne može imati parametre $\rightarrow$ najviše 1 po klasi
    \item Destruktor se piše kada treba osloboditi memoriju i ostale resurse
    \item Česta potreba $\rightarrow$ klasa sadrži članove koji su pokazivači ili reference na druge objekte\\
    Dobra praksa tad $\rightarrow$ metod za uništavanje delova, pozvan iz konstruktora
    \item Ponašanje kao i drugim metodima
\end{itemize}


\item Pozivanje destruktora
\begin{itemize}
    \item Implicitno se poziva na kraju životnog veka objekta
    \item Pri uništavanju dinamičkog objekte koristeći \inlinecode{delete}
    \item Pri uništavanju dinamičkog niza - u smeru opadajućih indeksa
    \item Redosled je uvek obrnut od konstruktora
    \item Eksplicitno pozivanje\\
    \inlinecode{X.$\sim$X(), px->$\sim$X(), this->$\sim$X()}
    \begin{itemize}
        \item[-] Ne preporučuje se, objekat nastavi da živi i posle ovoga
    \end{itemize}
    \item Posle izvršenja automatskog destruktora se oslobađa zauzeta memorija
\end{itemize}
\item Statički (zajednički) atributi
\begin{itemize}
    \item Pri stvaranju objekta klase $\rightarrow$ poseban komplet nestatičkih atributa
    \item Ključna reč - \inlinecode{static}
    \item Jedan primerak za celu klasu, svi objekti ga dele\\
    \inlinecode{static <tip> ime;}
\end{itemize}
\item Definisanje statičkog atributa
\begin{itemize}
    \item U klasi se samo deklariše
    \item Mora da se definiša na globalnom nivou
    \item Svi oblici inicijalizatora \checkmark{}
    \item Inicijalizacija
    \begin{enumerate}
        \item Pre prvog pristupa njemu
        \item Pre stvaranja objekta date klase
    \end{enumerate}
    \item Obraćanje \inlinecode{int <klasa>::X=5; // bez static}
    \item Ako se navede inicijalizator $\rightarrow$ 0
    \item \boxedimportant[ČUDNO]{Imenovana \textbf{celobrojna} konstanta može se definisati i u definiciji klase}
    \marginnote{Ima veze sa constexpr?}
   
\end{itemize}
\newpage
\item Statički i globalni podaci
\begin{itemize}
    \item Sličnosti
    \begin{enumerate}
        \item Trajni podaci $\rightarrow$ sličan životni vek
        \item Definicija na globalnom nivou
    \end{enumerate}
    \item Razlike
    \begin{enumerate}
        \item Statički atributi pripadaju klasi
        \item Doseg imena statičkog atributa je klasa
        \item Statičkim atributima je moguće ograničiti pristup
    \end{enumerate}
    \item Statički atribut ima sva svojstva globalnog statičkog podatka osim dosega imena i kontrole pristupa
    \item Smanjuje se potreba za globalnim objektima
\end{itemize}


\item Statički (zajednički) metodi
\begin{itemize}
    \item Funkcija klase, a ne svakog posebnog objekta
    \item Zajednički za sve objekte
    \item Primena
    \begin{enumerate}
        \item Opšte usluge
        \item Obrada statičkih atributa
    \end{enumerate}
    \item Deklarišu se dodavanjem \inlinecode{static} ispred deklaracije
    \item Svojstva globalnih funkcija osim dosega i kontrole pristupa
    \item Nemaju \inlinecode{this}
    \begin{enumerate}
        \item Ne mogu pristupati nestatičkim članovima direktnim imenovanjem
        \item Modifikatori \inlinecode{const} i ostali nemaju smisla
    \end{enumerate}
    \item Mogu pristupati nestatičkim članovima konkretnih objekata
    \begin{enumerate}
        \item Pristup preko parametra
        \item Pristup lokalnom objekta
        \item Pristup globalnom objektu
    \end{enumerate}
    \item Direktan pristup statičkim članovima\\
    \inlinecode{<klasa>::<ime\_funkcije>(argumenti);}
    \item Može se pozvati za konkretan objekat, ali izbegavati\\
    Levi operand tada samo nađe tip bez ikakvog izračunavanja
    \item Mogu se pozivati i pre stvaranja objekta klase
    \item Uslužna klasa $\rightarrow$ sve statički metodi, obrisanog ugrađenog konstruktora - kao biblioteka
\begin{center}
    \begin{adjustwidth}{-4cm}{-2.5cm}

    \begin{multicols}{2}
    \begin{lstlisting}
class X {
    static int x; // staticki atribut
    int y;
public:
    static int f(X); // staticki metod (deklaracija)
    int g();
};
int X::x=5; // definicija statickog atributa
int X::f(X x1){ // definicija statickog metoda
    int i=x; // pristup statickom atributu X::x
    int j=y;  // ERROR - X::y nije staticki
    int k=x1.y; // ovo moze;
    (x1++).x; // x1++ (ako je definisan post inkrement operator) se ne izracunava po Tartalji 
    return x1.x;  // i ovo moze, ali nije preporucljivo
} // izraz "x1" se ne izracunava 
    \end{lstlisting}
    \columnbreak
    \begin{lstlisting}

int X::g () {
    int i=x; // nestaticki metod moze da koristi
    int j=y; // i staticke i nestaticke atribute
    return j; // y je ovde this->y;
}
int main () {
    X xx;
    int p=X::f(xx); 
    // X::f moze neposredno, bez objekta;
    int q=X::g(); 
    // ERROR - za X::g mora konkretan objekat
    xx.g(); // ovako moze;
    p=xx.f(xx); // i ovako moze, ali nije preporucljivo
}

 
  
   
    
     
     
     
    \end{lstlisting}
    \end{multicols}
\end{adjustwidth}
\end{center}
 \newpage
    \begin{lstlisting}
// Zadatak koji se pojavio na kolokvijumu
class X {
public: static X* kreiraj () { return new X; }
private: X(); // Konstruktor je privatan
};
int main() {
    X x; // ERROR
    X* px=X::kreiraj(); // OK
}
 \end{lstlisting}

\end{itemize}
\item Prijatelji klasa
\begin{itemize}
    \item Kad je potrebno da klasa ima povlašćene korisnike koji mogu da pristupaju njenim privatnim članovima
    \item Povlašćene mogu biti
    \begin{enumerate}
        \item Funkcije
        \item Cele klase
    \end{enumerate}
    \item Nazivamo ih \textbf{prijateljima} - \textit{friends}
    \item Prijateljstvo, kao relacija između klasa
    \begin{enumerate}
        \item Ne nasleđuje se
        \item Nije simetrično \frownie
        \item Nije tranzitivno
    \end{enumerate}
    \item Reguliše isključivo pravo pristupa, a ne i oblast važenja i vidljivost identifikatora
\end{itemize}
\item Prijateljski funkcije
\begin{itemize}
    \item Nisu članice klasa ali imaju pristup privatnim članovima
    \item Mogu biti metode druge klase ili globalne funkcije
    \item Funkcija je prijateljska ako se u definiciji klase navede njena deklaracija ili definicija sa modifikatorom \inlinecode{friend}
    \item Klasa mora eksplicitno da naglasi prijateljstvo
    \item Ako u definiciji klase pišemo prijateljsku funkciju
    \begin{enumerate}
        \item I dalje nije članica klase iako je definišemo unutar nje
        \item Podrazumeva se da je \inlinecode{inline}
        \item Funkcija nema klasni doseg već doseg identifikatora klase
    \end{enumerate}
    \item Nevažno je pravo pristupa za \inlinecode{friend} funkciju
    \item Nema \inlinecode{this}
    \item Funkcija može biti prijatelj većem broju klasa istovremeno
    \begin{lstlisting}
class X {
    friend void g(int, X&); // prijateljska globalna funkcija
    friend void Y::h(); // prijateljski metod druge klase
    friend int o(X x){return x.i;}// definicija globalne f-je
    friend int p(){return i;} // ERROR - nema this
    int i;
public:
    void f(int ip) {i=ip;}
};
void g (int k, X &x) { x.i=k; }
int main () {
    X x; int j;
    x.f(5); // P preko metoda
    g(6,x); // Postavljanje preko prijateljske funkcije
    j=o(x); // Citanje preko prijateljske funkcije
}
    \end{lstlisting}
\end{itemize}
 \newpage
\item Prijateljske funkcije i metodi
\begin{itemize}
    \item Nekad je bolja prijateljska funkcija od metoda
    \item Metod mora da se pozove za objekat date klase, dok globalnoj funkciji možemo dostaviti i oblik drugog tipa\\
    Nemoguća konverzija skrivenog argumenta u metodu
    \item Pristup privatnim članovima više klasa - simetrično rešenje
    \item Nekad je jenotacija pogodnija\\
    \inlinecode{max(a,b)} ili \inlinecode{a.max(b)}
    \item Kad se preklapaju operator, često je jednostavnije definisati globalne operatorske funkcije nego metode
\end{itemize}
\item Prijateljske klase
\begin{itemize}
    \item Ako su svi metodi klase Y prijateljske funkcije klase X, onda je Y prijateljska klasa (\textit{friend class}) klasi X\\\\
    \inlinecode{class X \{}\\
    \hspace*{1cm}\inlinecode{friend Y; // Ako je klasa Y definisana ili deklarisana}\\
    \hspace*{1cm}\inlinecode{friend class Z; // Ako Z nije ni definisana ni deklarisana}\\
    \inlinecode{\};}
    \item Svi metodi klase Y pristupaju privatnim članovima klase X
    \item Prijateljske klase se često koriste kad neke dve klase imaju tesnu vezu
\end{itemize}
\item Ugnježdene klase
\begin{itemize}
    \item Klase mogu da se deklarišu ili definišu unutar definicije druge klase
    \item Koristi se kada neki tip semantički pripada samo datoj klasi
    \item Povećava čitljivost i smanjuje potrebu za globalnim tipovima
    \item Unutar definicije klase se mogu navesti i definicije nabrajanja \inlinecode{enum} i tipova \inlinecode{typedef}
    \item Ugnježdena klasa se nalaziu dosegu imena okružujuće klase (izvan nje pristup samo sa \inlinecode{::})
    \item Iz okružujuće klase u ugnježdenu \inlinecode{., ->, ::}
    \item Doseg imena okružujuće klase O se proteže na ugnježdenu klasu U\\
    Pristup iz U do članova O samo sa \inlinecode{., ->}
    \item U ugnježdenoj klasi mogu direktno da se koriste identifikatori
    \begin{enumerate}
        \item \boxedimportant[]{Tipova iz okružujuće klase $\longleftarrow$ Samo od konkretnog objekta}
        \item Konstanti tipa nabrajanja okružujuće klase
        \item Statički članovi okružujuće klase
    \end{enumerate}
    \item \boxedimportant[Zar nije obrnuto?]{Ovo važi ako ime nije sakriveno imenom člana \textbf{ugnježdene} klase}\\\\
    \inlinecode{<id\_okružujuće>::<id\_ugnježdene>::<id\_statičkog>}
    \item Ugnježdena klasa je implicitno prijatelj okružujuće
    \item Okružujuća klasa nije prijatelj ugnježdene \frownie $\longleftarrow$ ugnježdena klasa
    \begin{lstlisting}
int x,y;
class Spoljna {
public:
    int x; static int z;
    class Unutrasnja {
        void f(int i, Spoljna *ps) {
            x=i; // ERROR - nepoznat objekat klase Spoljna
            Spoljna::x=i; // ERROR - isti uzrok
            z=i; // pristup statickom clanu Spoljna
            ::x=i; // pristup globalnom x;
            y=i; // pristup globalnom y;
            ps->x=i; // pristup Spoljna::x objekta *ps; 
        }
    };
};
Unutrasnja u; // ERROR
Spoljna::Unutrasnja u; // OK
    \end{lstlisting}
\end{itemize}

\item Strukture
\begin{itemize}
    \item Struktura je klasa kod koje su svi članovi podrazumevano javni\\
    Može se menjati eksplicitnim korišćenjem \inlinecode{public:} i \inlinecode{private:}
    \item C++ struktura može imati i metode
    \item Strukture se koriste za definisanje struktuiranih podataka koje ne predstavljaju apstrakciju i generalno nemaju značajnijih operacija
    \item Tipično imaju samo konstruktor, uz eventualni destruktor
\end{itemize}
\item Lokalne klase 
\begin{itemize}
    \item Definišu se unutar funkcija
    \item Identifikator ima doseg od deklaracije do kraja bloka u kom je deklarisan
    \item Unutar klase dozvoljeno je korišćenje iz okružujućeg dosega
    \begin{enumerate}
        \item Identifikatora tipova
        \item Konstanti tipa nabrajanja
        \item Trajnih podataka (statičkih atributa, statičkih lokalnih i globalnih)
        \item Spoljašnjih (\inlinecode{extern}) podataka i funkcija
    \end{enumerate}
    \item Metodi lokalne klase moraju da se definišu unutar definicije klasa
    \item Lokalna klasa ne može da ima statičke atribute, dok može imati statičke metode
    \begin{lstlisting}
int x;
void f() {
    static int s;
    int x;
    extern int g();
    class Lokalna {
    public:
        int h (){ return x; } // ERROR - x je automatska prom.
        int j (){ return s; } // OK: s je staticka promenljiva
        int k (){ return ::x; } // OK: x je globalna promenljiva
        int l (){ return g(); } // OK: g() je spoljasnja funkcija
    };
}
Lokalna *p = 0; // ERROR - nije u dosegu
    \end{lstlisting}
\end{itemize}
\item Pokazivači na članove klase
\begin{itemize}
    \item Dodelom vrednosti pokazivaču na članove klase označi se nei član klase\\
    Kao pokazivačka aritmetika indeksa u nizu
    \item Deklaracija\\
    \inlinecode{<tip\_člana><klasa>::*<identifikator>}
    \item Formiranje adrese\\
    \inlinecode{<identifikator> = \&<klasa>::<član>}
    \item Pristup\\
    \inlinecode{<objekat>.*<identifikator>}\\
    \inlinecode{<pokazivač\_na\_objekat>->*<identifikator>}
    \item \inlinecode{.*} i \inlinecode{->*} imaju prioritet 14 i asocijativnost sleva na desno
    \begin{lstlisting}
class Alfa {... public: int a, b; };

int Alfa::*pc;// pc je pokazivac na int clanove klase Alfa
Alfa alfa,*beta;
beta=&alfa;

pc=&Alfa::a;// pc pokazuje na clanove a objekata klase Alfa
alfa.*pc = 1; // alfa.a=1;
beta->*pc = 1; // beta->a=1;

pc=&Alfa::b;// pc pokazuje na clanove b objekata klase Alfa
alfa.*pc = 2; // alfa.b=2;
beta->*pc = 2; // beta->b=2;
    \end{lstlisting}
\end{itemize}
\end{xitemize}
\begin{center}
    SI KOLOKVIJUM 1
\end{center}
\noindent\makebox[\linewidth]{\rule{\paperwidth}{0.4pt}}

\end{document}
